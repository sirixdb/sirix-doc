\documentclass[10pt,twoside,a4paper,twocolumn,abstracton]{scrartcl}
\setlength{\oddsidemargin}{-0.4mm} % 25 mm left margin
\setlength{\evensidemargin}{\oddsidemargin}
\setlength{\textwidth}{160mm}      % 25 mm right margin
\setlength{\topmargin}{-5.4mm}     % 20 mm top margin
\setlength{\headheight}{5mm}
\setlength{\headsep}{5mm}
\setlength{\footskip}{10mm}
\setlength{\textheight}{237mm}     % 20 mm bottom margin

\usepackage[utf8]{inputenc}
\usepackage{lmodern}
\usepackage[pdfpagelabels=false,pdfborder={0 0 0}]{hyperref}
\usepackage{graphicx}

\newenvironment{keywords}%
   {\begin{trivlist}\item[]{\bfseries\sffamily Keywords:}\ }
   {\end{trivlist}}

\begin{document}

\title{Sirix - Beyond Versioning of Persistent Trees}
\subtitle{Storing, Querying and Analysing Differences}
\author{Johannes Lichtenberger \thanks{email: \texttt{lichtenberger.johannes@gmail.com}}}

\maketitle

\begin{abstract}
Nowadays, disk space is cheap and versioning becomes increasingly interesting. Besides, the days of mechanical disks are numbered and flash drives will certainly predominate in the future. We propose a system which takes full advantage of fast random reads and sequential log-structured writes of flash drives to store fine grained temporal trees. Retaining not only the most recent version but also snapshots of the past facilitates the analysis of temporal patterns.
\end{abstract}

\begin{keywords}
Versioning, Git, SVN, Mercurial, Revision, Diffing, XML, JSON, XQuery, XPath, Storage, Visual Analytics
\end{keywords}

\section{Introduction}
Storage space follows Moore's Law and doubles approximately every two years. It becomes increasingly tempting to store not only the current version, but all past snapshots of data sets. Hence, it is possible to go back in time in the advent of any kind of failures or unintentional modifications. Furthermore analytical tasks to detect patterns in temporal data are supported, due to the availability of each stored snapshot. Moreover, semi-structured data in the form of tree-structures is very common either in the serialized form as XML-dialects or in the JSON format.

\subsection{Problem Statement}
By retaining old versions of fine grained trees we face the problem of having to cope with increasingly large data sets. Storing new versions of collections of whole trees is very expensive and inefficient. The storage layer thus has to focus on an effective and efficient page-to-block management for the on-disk storage. In order to support fast queries while not restricting the write-performance suitable versioning strategies have to be found. Besides, query languages as for instance XPath or its big brother XQuer have no notion of temporal aspects and thus have to be extended to allow navigation not only in space, but also in time.
\subsection{Contribution}

\section{Conclusion}

\section{Acknowledgements}

%\begin{equation}
%    \label{simple_equation}
%    \alpha = \sqrt{ \beta }
%\end{equation}

%\subsection{Subsection Heading Here}
%Write your subsection text here.

%\begin{figure}
%    \centering
%    \includegraphics[width=3.0in]{myfigure}
%    \caption{Simulation Results}
%    \label{simulationfigure}
%\end{figure}

%\section{Conclusion}
%Write your conclusion here.

\end{document}
